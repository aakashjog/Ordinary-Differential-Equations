\documentclass[fleqn, a4paper, 12pt, oneside]{amsart}
\usepackage{exsheets}
\usepackage{tasks}
\usepackage{amsmath, amssymb, amsthm} %standard AMS packages
\usepackage{marginnote} %marginnotes
\usepackage{gensymb} %miscellaneous symbols
\usepackage{commath} %differential symbols
\usepackage{xcolor} %colours
\usepackage{cancel} %cancelling terms
\usepackage{siunitx} %formatting units
\usepackage{tikz, pgfplots} %diagrams
\usetikzlibrary{calc, hobby, patterns, intersections}
\usepackage{graphicx} %inserting graphics
\usepackage{hyperref} %hyperlinks
\usepackage{datetime} %date and time
\usepackage{ulem} %underline for \emph{}
\usepackage{xfrac} %inline fractions
\usepackage{enumerate, enumitem} %numbered lists
\usepackage{float} %inserting floats

\newcommand\numberthis{\addtocounter{equation}{1}\tag{\theequation}} %adds numbers to specific equations in non-numbered list of equations

\newcommand{\AxisRotator}[1][rotate=0]{
	\tikz [x=0.25cm,y=0.60cm,line width=.2ex,-stealth,#1] \draw (0,0) arc (-150:150:1 and 1);%
} %rotation symbols on axes

\theoremstyle{definition}
\newtheorem{example}{Example}
\newtheorem{definition}{Definition}

\theoremstyle{theorem}
\newtheorem{theorem}{Theorem}

\newcommand{\curl}{\mathrm{curl\,}}

\makeatletter
\@addtoreset{section}{part} %resets section numbers in new part
\makeatother

\renewcommand{\thesubsection}{(\arabic{subsection})}
\renewcommand{\thesection}{(\arabic{section})}

%section headings on left
\makeatletter
\def\specialsection{\@startsection{section}{1}%
	\z@{\linespacing\@plus\linespacing}{.5\linespacing}%
	%  {\normalfont\centering}}% DELETED
	{\normalfont}}% NEW
\def\section{\@startsection{section}{1}%
	\z@{.7\linespacing\@plus\linespacing}{.5\linespacing}%
	%  {\normalfont\scshape\centering}}% DELETED
	{\normalfont\scshape}}% NEW
\makeatother

%forces newline after subsection
\makeatletter
\def\subsection{\@startsection{subsection}{3}%
	\z@{.5\linespacing\@plus.7\linespacing}{.1\linespacing}%
	{\normalfont\itshape}}
\makeatother

\settasks{counter-format = tsk[a])}

\SetupExSheets{solution/print = true}

\makeatletter
\@addtoreset{question}{part} %resets section numbers in new part
\makeatother

%opening
\title{Ordinary Differential Equations : Assignment 3}
\author
{
	Aakash Jog\\
	ID : 989323563
}
\date{\formatdate{4}{5}{2015}}

\begin{document}
	
\maketitle
%\setlength{\mathindent}{0pt}

\part{Lipschitz Continuous Functions}

\begin{question}
	Prove that any Lipschitz function is continuous.
\end{question}

\begin{solution}
	Let $f(x)$ be a Lipschitz function.\\
	Therefore,
	\begin{align*}
		|f(x_1) - f(x_2)|                                           & \le k |x_1 - x_2| \\
		\therefore \left| \frac{f(x_1) - f(x_2)}{x_1 - x_2} \right| & \le k             \\
		\therefore \dod{f(x)}{x}                                    & \le k
	\end{align*}
	Therefore, as the derivative of $f(x)$ exists, $f(x)$ must be continuous.
\end{solution}

\begin{question}
	Prove that any continuously differentiable function on an interval $[a,b]$ is Lipschitz there.
\end{question}

\begin{solution}
	Let the function be $f(x)$.\\
	As $f(x)$ is continuously differentiable, $f'(x)$ exists on $[a,b]$ and is never infinite.\\
	Therefore,
	\begin{align*}
		\left| \frac{f(x_1) - f(x_2)}{x_1 - x_2} \right| & \le k \\
		\therefore | f(x_1) - f(x_2)|                    & \le k |x_1 - x_2|
	\end{align*}
	Therefore, $f(x)$ is Lipschitz on $[a,b]$.
\end{solution}

\begin{question}
	Show that $f(x) = |x|$ is a Lipschitz continuous function on the whole real line.
\end{question}

\begin{solution}
	\begin{align*}
		|f(x_1) - f(x_2)|            & = \left| |x_1| - |x_2| \right| \\
		\intertext{By triangle inequality theorem,}
		\left| |x_1| - |x_2| \right| & \le |x_1 - x_2|                \\
		\therefore |f(x_1) - f(x_2)  & \le |x_1 - x_2|
	\end{align*}
	Therefore, $f(x)$ is Lipschitz on $\mathbb{R}$.
\end{solution}

\begin{question}
	Show that $g(x) = \sqrt{x}$ is not a Lipschitz continuous function on $[0,1]$.
\end{question}

\begin{solution}
	If possible let $g(x) = \sqrt{x}$ be a Lipschitz continuous function on $[0,1]$.\\
	Therefore, if $x_1 \neq x_2$,
	\begin{align*}
		\left| \sqrt{x_1} - \sqrt{x_2} \right|                                                                                              & \le k |x_1 - x_2| \\
		\therefore \left| \sqrt{x_1} - \sqrt{x_2} \right| - k \left| \sqrt{x_1} - \sqrt{x_2} \right| \left| \sqrt{x_1} + \sqrt{x_2} \right| & \le 0             \\
		\therefore \left| \sqrt{x_1} - \sqrt{x_2} \right| \left( 1 - k \left| \sqrt{x_1} + \sqrt{x_2} \right| \right)                       & \le 0             \\
		\therefore \left| \sqrt{x_1} - \sqrt{x_2} \right|                                                                                   & \le \frac{1}{1 - k \left| \sqrt{x_1} + \sqrt{x_2} \right|}
	\end{align*}
	Therefore, if $x_1 \neq x_2$, $x_1 \to \infty$, $x_2 \to \infty$, then $|f(x_1) - f(x_2)|$ does not exist.\\
	Therefore, the function is not Lipschitz on $[0,1]$.
\end{solution}

\begin{question}
	Show that $h(x) = x^2$ is Lipschitz continuous function on any closed interval $[a,b]$, but it is not globally Lipschitz continuous on the whole real line.
\end{question}

\begin{solution}
	$h'(x) = 2 x$ is continuous on any closed interval $[a,b]$.\\
	Therefore, $h(x) = x^2$ is Lipschitz on any closed interval $[a,b]$.\\
	For $h(x)$ to be Lipschitz, assuming $x_1 \neq x_2$,
	\begin{align*}
		|{x_1}^2 - {x_2}^2|                & \le k |x_1 - x_2| \\
		\therefore |x_1 + x_2| |x_1 - x_2| & \le k |x_1 - x_2| \\
		\therefore |x_1 + x_2|             & \le k
	\end{align*}
	If the interval is $\mathbb{R}$, there cannot be one particular $k$ which satisfies the above inequality.\\
	Therefore, $h(x)$ is not Lipschitz on $\mathbb{R}$.
\end{solution}

\begin{question}
	Is the function $k(x) = \sqrt{x^2 + 5}$ continuous on the whole real line?
\end{question}

\begin{solution}
	If possible, let $k(x)$ be Lipschitz on the whole real line.\\
	Therefore,
	\begin{align*}
		\left| \sqrt{{x_1}^2 + 5} - \sqrt{{x_2}^2 + 5} \right|                                & \le k |x_1 - x_2|                                                        \\
		\therefore \left| {x_1}^2 + 5 - {x_2}^2 - 5 \right|                                   & \le k |x_1 - x_2| \left| \sqrt{{x_1}^2 + 5} + \sqrt{{x_2}^2 + 5} \right| \\
		\therefore |x_1 + x_2|                                                                & \le \left| \sqrt{{x_1}^2 + 5} + \sqrt{{x_2}^2 + 5} \right|               \\
		\therefore \frac{|x_1 + x_2|}{\left| \sqrt{{x_1}^2 + 5} + \sqrt{{x_2}^2 + 5} \right|} & \le k                                                                    \\
		\intertext{As $|x_1 + x_2| < \left| \sqrt{{x_1}^2 + 5} + \sqrt{{x_2}^2 + 5} \right|$,}
		1                                                                                     & \le k
	\end{align*}
	Therefore, $k(x)$ is Lipschitz continuous on the whole real line.
\end{solution}

\part{Uniqueness Conditions}

\begin{question}
	Find a solution for the initial value problem
	\begin{align*}
		y'   & = \frac{|\sin y|}{y} \\
		y(1) & = \pi
	\end{align*}
	and show it is unique.
\end{question}

\begin{solution}
	The function $f(x) = \pi$ satisfies the set of equations.\\
	~\\
	Let $y_1$, $y_2$ be two solutions to the initial value problem.\\
	\begin{align*}
		|y_1 - y_2|            & \le \int\limits_{0}^{x} \left| \frac{|\sin y_1|}{x'} - \frac{|\sin y_2|}{x_2} \right| \dif x' \\
		\therefore |y_1 - y_2| & \le k \int\limits_{0}^{k} |y_1 - y_2| \dif x'
	\end{align*}
	Let
	\begin{align*}
		a(x)             & = \int\limits_{0}^{x} |y_1 - y_2| \dif x' \\
		\therefore a'(x) & = |y_1 - y_2|
	\end{align*}
	Therefore,
	\begin{align*}
		a'(x)                                    & \le k a(x) \\
		\therefore \left( a(x) e^{-k x} \right)' & \le 0      \\
		\therefore a(x)                          & = 0        \\
		\therefore y_1                           & = y_2
	\end{align*}
	Therefore the solution is unique.
\end{solution}

\begin{question}
	Consider the initial value problem
	\begin{align*}
		y'   & = \frac{-t + (t^2 + 4 y)^{\frac{1}{2}}}{2} \\
		y(2) & = -1
	\end{align*}
	\begin{enumerate}
		\item
			Verify that both $y_1 = 1 - t$ and $y_2 = -\frac{t^2}{4}$ are solutions of the initial value problem.
			Where are these solutions valid?
		\item
			Explain why the existence of the two solutions of the given problem does not contradict the uniqueness part of the existence and uniqueness theorem.			
	\end{enumerate}
\end{question}

\begin{solution}
	\begin{enumerate}[leftmargin = *]
		\item 
			\begin{align*}
				{y_1}'(t) & = -1
			\end{align*}
			Substituting $y$ in $y'$,
			\begin{align*}
				y' & = \frac{-t + (t^2 + 4 - t)^{\frac{1}{2}}}{2} \\
                                   & = \frac{-t + (t - 2)^2}{2}                   \\
                                   & = -1
			\end{align*}
			\begin{align*}
				y_1(2) & = 1 - 2 \\
                                       & = -1
			\end{align*}
			Therefore $y_1 = 1 - t$ is a solution of the initial value problem.\\
			It is valid in $\mathbb{R}$.
			~\\
			\begin{align*}
				{y_2}'(t) & = -\frac{t}{2}
			\end{align*}
			Substituting $y$ in $y'$,
			\begin{align*}
				y' & = \frac{-t + (t^2 - t^2)^{\frac{1}{2}}}{2} \\
                                   & = -\frac{t}{2}
			\end{align*}
			\begin{align*}
				y_2(2) & = -\frac{2^2}{4} \\
                                       & = 1
			\end{align*}
			Therefore $y_2 = -\frac{t^2}{4}$ is a solution of the initial value problem.\\
			Due to the square root, it is valid in $[2,\infty)$.
		\item
			As the function is not Lipschitz in $y$, the existence and uniqueness theorem is not applicable to it.
			Therefore, it does not contradict the theorem.
	\end{enumerate}
\end{solution}

\part{Picard Approximations}

\begin{question}
	Consider the following initial value problems
	\begin{enumerate}[label = (\alph*)]
		\item $y' = 2 (y + 1)$, $y(0) = 0$
		\item $y' = y + 1 - t$, $y(0) = 0$
	\end{enumerate}
	\begin{enumerate}
		\item Find the Picard approximations $\varphi_n(t)$ for the solution of the initial value problems for an arbitrary $n$.
		\item Express $\lim\limits_{n \to \infty} \varphi_n(t)$ in terms of elementary functions.
		\item Solve the initial value problems using order 1 techniques and compare your results.
	\end{enumerate}
\end{question}

\begin{solution}
	\begin{enumerate}[label = (\alph*), leftmargin = *]
		\item
			\begin{enumerate}[leftmargin = *]
				\item
					\begin{align*}
						\varphi_0(t)            & = 0                                                                  \\
						\therefore \varphi_1(t) & = 0 + \int\limits_{0}^{t} 2 (1 + 0) \dif x                           \\
                                                                        & = 2 t
						\therefore \varphi_2(t) & = 0 + \int\limits_{0}^{t} 2 (1 + 2 x) \dif x                         \\
                                                                        & = 2 t + \frac{4 t^2}{2}                                              \\
						\therefore \varphi_3(t) & = 0 + \int\limits_{0}^{t} 2 \left( 1 + 2 x + \frac{4 x^2}{4} \right) \\
                                                                        & = 2 t + \frac{4 x^2}{2} + \frac{8 t^3}{6}                            \\
                                                                        & \vdots                                                               \\
						\therefore \varphi_n(t) & = \sum\limits_{r = 1}^{n} \frac{(2 t)^r}{r!}
					\end{align*}
				\item
					\begin{align*}
						\lim\limits_{n \to \infty} \varphi_n(t) & = \lim\limits_{n \to \infty} \sum\limits_{r = 1}^{n} \frac{(2 t)^r}{r!} \\
                                                                                        & = e^{2 t} - 1
					\end{align*}
				\item
					\begin{align*}
						\dod{y}{x}                      & = 2 (y + 1) \\
						\therefore \frac{\dif y}{y + 1} & = 2 \dif t  \\
						\therefore y + 1                & = c e^{2 t} \\
						\intertext{Substituting initial conditions, $c = 1$. Therefore,}
						y + 1                           & = e^{2 t}   \\
						\therefore y                    & = e^{2 t} - 1
					\end{align*}
			\end{enumerate}
		\item
			\begin{enumerate}[leftmargin = *]
				\item
					\begin{align*}
						\varphi_0(t)            & = 0                                                                       \\
						\therefore \varphi_1(t) & = \int\limits_{0}^{t} (0 + 1 - x) \dif x                                  \\
                                                                        & = t - \frac{t^2}{2}                                                       \\
						\therefore \varphi_2(t) & = \int\limits_{0}^{t} \left( 0 + 1 - x + x - \frac{x^2}{2} \right) \dif x \\
                                                                        & = t - \frac{t^3}{6}                                                       \\
                                                                        & \vdots                                                                    \\
						\therefore \varphi_n(t) & = t - \frac{t^{n + 1}}{(n + 1)!}
					\end{align*}
				\item
					\begin{align*}
						\lim\limits_{n \to \infty} \varphi_n(t) & = \left( t - \frac{t^{n + 1}}{(n + 1)!} \right) \\
                                                                                        & = t
					\end{align*}
				\item
					\begin{align*}
						\dod{y}{t}                & = y + 1 - t \\
						\therefore \dod{y}{t} - y & = 1 - t     \\
						\therefore y              & = t + c
						\intertext{Substituting initial conditions, $c = 0$. Therefore,}
						y                         & = t
					\end{align*}
			\end{enumerate}
	\end{enumerate}
\end{solution}

\begin{question}
	Calculate the Picard approximations $\varphi_1(t)$, $\varphi_2(t)$, $\varphi_3(t)$ for the initial value problem $y' = t^2 + y^2$, $y(0) = 0$.
\end{question}

\begin{solution}
	\begin{align*}
		\varphi_0(t)            & = 0                                                                                               \\
		\therefore \varphi_1(t) & = \int\limits_{0}^{t} (x^2 + 0^2) \dif x                                                          \\
                                        & = \frac{t^3}{3}                                                                                   \\
		\therefore \varphi_2(t) & = \int\limits_{0}^{t} \left( x^2 + \left( \frac{x^3}{3} \right)^2 \right) \dif x                  \\
                                        & = \frac{t^3}{3} + \frac{t^7}{3^2 \cdot 7}                                                         \\
		\therefore \varphi_3(t) & = \int\limits_{0}^{t} \left( x^2 + \left( \frac{x^3}{3} + \frac{x^7}{63} \right)^2 \right) \dif x \\
                                        & = \frac{t^3}{3} + \frac{t^7}{63} + \frac{2}{3^2 \cdot 7} \cdot \frac{t^{11}}{11} + \frac{1}{\left( 3^2 \cdot 7 \right)^2} \cdot \frac{t^{15}}{15}
	\end{align*}
\end{solution}

\begin{question}
	Use the pattern
	\begin{align*}
		\varphi_0(t) & = x(0) & \varphi_{i + 1}(t) & = x(0) + \int\limits_{0}^{t} f_1 \left( s, \varphi_i(x), \psi_i(s) \right) \dif s \\
		\psi_0(t)    & = y(0) & \psi_{i + 1}(t)    & = y(0) + \int\limits_{0}^{t} f_2 \left( s, \varphi_i(x), \psi_i(s) \right) \dif s
	\end{align*}
	to find the first four Picard approximations for the solution of the initial value problem
	\begin{align*}
		\dod{x}{t} & = t + y   & x(0) & = 0 \\
		\dod{y}{t} & = t - x^2 & y(0) & = 1
	\end{align*}
\end{question}

\begin{solution}
	\begin{align*}
		\varphi_0(t) & = 2 \\
		\psi_0(t)    & = 1
	\end{align*}
	Therefore,
	\begin{align*}
		\varphi_1(t) & = 2 + \int\limits_{0}^{t} (t' + 1) \dif t'   \\
                             & = 2 + t + \dfrac{t^2}{2}                     \\
		\psi_1(t)    & = 1 + \int\limits_{0}^{t} (t' - 2^2) \dif t' \\
                             & = 1 - 4 t + \frac{t^2}{2}
	\end{align*}
	Therefore,
	\begin{align*}
		\varphi_2(t) & = 2 + \int\limits_{0}^{t} \left( t' + 1 - 4 t' + \frac{t'^2}{2} \right) \dif t'                \\
                             & = 2 + t - \frac{3 t^2}{2} + \frac{t^3}{6}                                                      \\
		\psi_2(t)    & = 1 + \int\limits_{0}^{t} \left( t' - \left( 2 + t' + \frac{t'^2}{2} \right)^2 \right) \dif t' \\
                             & = 1 + \left( -\frac{t^5}{20} - \frac{t^4}{4} - t^3 - \frac{3 t^2}{2} - 4 t \right)             \\
	\end{align*}
\end{solution}

\end{document}
